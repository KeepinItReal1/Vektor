\subsection*{Pastabos}


\begin{DoxyItemize}
\item Truksta releasu
\item Truksta cmake
\item Pataisyta ir Jūs sakėt \char`\"{}labai gerai\char`\"{} \+:D
\end{DoxyItemize}

\subsection*{Privalumai}


\begin{DoxyItemize}
\item Veikia greitai Templeitai i tema panaudoti
\end{DoxyItemize}

\section*{Objektinis\+Programavimas}

\section*{Naudojimosi instrukcija}

\subsection*{Meniu atrodo taip\+:}

\tabulinesep=1mm
\begin{longtabu} spread 0pt [c]{*{2}{|X[-1]}|}
\hline
\rowcolor{\tableheadbgcolor}\textbf{ Funkcija  }&\textbf{ Trumpas aprašyma   }\\\cline{1-2}
\endfirsthead
\hline
\endfoot
\hline
\rowcolor{\tableheadbgcolor}\textbf{ Funkcija  }&\textbf{ Trumpas aprašyma   }\\\cline{1-2}
\endhead
0  &to end the program.   \\\cline{1-2}
1  &to create a file.   \\\cline{1-2}
2  &to read a file with vector/list/deque.   \\\cline{1-2}
3  &to test vector/list/deque with 5 files.   \\\cline{1-2}
4  &to test vector/list/deque with 5 files differently(2 holders).   \\\cline{1-2}
\end{longtabu}


{\bfseries Arba\+:} \href{https://gyazo.com/9c2eff8231fa73817b48c713a1f55d1c}{\tt Meniu nuotrauka}

\subsection*{0}

{\itshape Pasirinkus {\bfseries 0} programa kaip ir tikitės baigsis, parodys kiek laiko programa buvo aktyvi.} \href{https://gyazo.com/092a6b2fd577d833bff496f68325c0ff}{\tt {\bfseries Iliustracija \#0}}

\subsection*{1}

{\itshape Pasirinkus {\bfseries 1} Jūs busite paklaustas įvesti failo pavadinimą, kiek faile norite turėti įrašų, tada galėsite pasirinkti ar generuoti egzamino pažymį ir namų darbų įvertinimus automatiškai ar juos įvesti Jums pačiam.} \href{https://gyazo.com/0f9ab18b79c812de4335e95fd6494cab}{\tt {\bfseries Iliustracija \#1}}

\subsection*{2}

{\itshape Pasirinkus {\bfseries 2} Jūs busite paklaustas įvesti failo, kurį norite atidaryti, pavadinimą bei galėsite pasirinkti su kuriuo laikikliu jį atidaryti (vector/list/deque).} \href{https://gyazo.com/681e2257febd29db1154daa58b56ecb5}{\tt {\bfseries Iliustracija \#2}}

\subsection*{3}

{\itshape Pasirinkus {\bfseries 3} programa automatiškai generuos 5 (10, 10$^\wedge$2, 10$^\wedge$3, 10$^\wedge$4, 10$^\wedge$5 įrašų dydžio) failus, juos nuskaitys su vienu iš konteinerių, surūšiuos \char`\"{}pagr\char`\"{} konteinerio įrašus į 2 tipus (\char`\"{}lievi\char`\"{} ir \char`\"{}geri\char`\"{}) pagal jų galutinį balą, apskaičiuotą pagal namų darbų vidurkius. Tai padariusi programa gražins nuskaitytų failų pavadinimus ir per kiek laiko įvyko nuskaitymas bei rūšiavimas.} \href{https://gyazo.com/a7fe7bea97258ca5a3c0ed2d4074b370}{\tt {\bfseries Iliustracija \#3}}

\subsection*{4}

{\itshape Pasirinkus {\bfseries 4} programa automatiškai generuos 5 (10, 10$^\wedge$2, 10$^\wedge$3, 10$^\wedge$4, 10$^\wedge$5 įrašų dydžio) failus, juos nuskaitys su vienu iš laikiklių, surūšiuos į 2 tipus (\char`\"{}lievi\char`\"{} ir \char`\"{}pagr\char`\"{}) pagal jų galutinį balą, apskaičiuotą pagal namų darbų vidurkius. Skirtingai nei pasirinkus {\bfseries 3} \char`\"{}pagr\char`\"{} konteineryje bus laikomi ne visi, o tik \char`\"{}geri\char`\"{} įrašai. Tai padariusi programa gražins nuskaitytų failų pavadinimus ir per kiek laiko įvyko nuskaitymas bei rūšiavimas.} \href{https://gyazo.com/141621d4d6ed0852214b934ba4f99b6a}{\tt {\bfseries Iliustracija \#4}} 